\section{Executive Summary}
Team Watts has a viable business model in which a trained image classification model becomes the principal component of a grab-and-go cafeteria automation system. There are over 3000 malls in Canada that would be targeted as potential customers in Canada alone, to say nothing of hospitals, sports facilities, and schools. The project is financially sustainable and highly profitable, returning 75 share percent of revenue in its first year. The ensemble model method is performing at 70\% accuracy on a single target, and 100\% on a double target. The model is still under active development to deal with overfitting. The data used in the demonstration described herein was of mixed quality and will be replaced by bespoke data collected from customer sites.

\section{Business Case}
\subsection{Problem Definition}
Time has become a significant factor for many individuals throughout this era regardless of career. Much of that time is spent eating to stay healthy and energized. However, COVID-19 triggered a trend in North America wherein employees are walking out of public-facing jobs such as restaurant table service and retail\footcite{petro}. In-house, fast-food, and cafeteria-style food services are feeling the pressure. The slower line-ups at the register and the rising price of food affect customer experience. The team has found a 20\% increase in customer preference for contactless operations post-pandemic\footcite{neal}. Should they not react to the labor shortages and customer preferences, food services will have a more challenging time competing against the convenience of home delivery and the lower prices of home-prepared meals.

Team Watts has decided to build an AI image classification system that could be deployed in cafeterias in various domains such as workplaces, malls, hospitals, schools, and entertainment facilities. This image classification system would change consumers’ lives by making grab-and- go meals more straightforward and more convenient. It proposes a tangible product, “Watts’ Food Detector,” that permits cafeteria customers to bypass checkout lineups. The concept is an amalgamation of, and an improvement on, the self-checkout scanners now used in grocery stores and pharmacies, the Sobey’s Smart Cart device, the MyFitnessPal app, and the customer stamp- card concept used by the Marché-Mövenpick and Vapiano chains in Germany. The goal is to increase traffic volume during peak hours and influence repeat sales through a positive customer experience.

